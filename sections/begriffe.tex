\section{Begriffe}
\subsection{Klasse}
	Eine Klasse ist eine Vorlage für ein Objekt.\\
	\begin{tabular}{p{3.5cm}p{14.5cm}}
		attribut & beschreibt Daten die von den Objekten der Klasse angenommen werden
		können.\\
		& alle Objekte einer Klasse haben dieselben Attribute, können
		aber unterschiedliche Attributwerte haben.\\
		klassenattribut & wenn nur ein
		Attributwert für alle Objekte einer Klasse existiert.\\
		& sie existieren auch, wenn es zu einer Klasse noch keine Objekte gibt.\\
		operation & eine Funktion die auf alle Attributwerte eines Objekts zugriff
		hat\\
		klassenoperation & eine Operation, die der jeweiligen Klasse zugeordnet ist\\
		& kann nicht auf ein einzelnes Objekt der Klasse angewendet werden\\
		verhalten & das Verhalten der Klasse ist die Menge aller operationen\\
		abstrakte & von einer Abstrakten Klasse können keine Objekte erzeugt werden\\
	\end{tabular}
\subsection{Objekt}
	Ein Objekt wird aus einer Klasse erzeugt, ist also ein Exemplar einer Klasse.\\
	\begin{tabular}{p{3.5cm}p{14.5cm}}
		Zustand & bestimmt durch seine Attributwerte und seine Objektbeziehungen zu
		anderen Objekten\\
		Verhalten (behavior) & die beobachtbaren Effekte aller Operationen\\
		& bestimmt durch die Operationsaufrufe, auf die diese Klasse bzw. deren Objekte
		reagieren.\\
		Objektidendität & jedes Objekt besitzt eine, sind unique\\
	\end{tabular}
\subsection{Komponente}
Ist ein Softwarebaustein, der über klar definierte Schnittstellen Verhalten
(Funktionalität) bereitstellt.
interface
component

\subsection{Assoziation}
	modelliert Objektbeziehungen zwischen Objekten einer oder mehrerer Klassen. Jede
	Assoziation wird durch Mutliplizitäten, einen optionalen Namen oder Rollennamen
	beschrieben.\\
	\begin{tabular}{p{3.5cm}p{14.5cm}}
		ternäre Assoziation & Assoziation zwischen 3 Objekten.\\
		n-äre Assoziation & zwischen n Objekten\\
		reflexive Assoziation & verbindet zwei Objekte einer Klasse\\
		binäre Assoziation & verbindet zwei Objekte \\
		Assoziationsklasse & besitzt die Eigenschaften einer Assoziation und einer
		Klasse \\
		Aggregation & Sonderfall der Ass. "`ist Teil von"' oder "`besteht aus"'.\\
		Komposition & besondere Form der Aggregation. Beim Löschen müssen auch alle
		Teile gelöscht werden. Jedes Teil kann nur zu einem Ganzen gehören.\\
		Navigationsrichtung & das erzeugen des einen erzwingt die Erzeugung des
		anderen\\
		Multiplizität & bezeichnet die Wichtigkeit der Assoziation. Bezeichnet die
		Anzahl der an der Assoziation beteiligten Objekte.\\
		abgeleitete Assoziation &
		die Abhängigkeiten sind bereits durch andere Assoziationen beschrieben worden\\
		Assoziationname & beschreibt im Allgemeinen nur eine Richtung der
		Assoziation. kann fehlen wenn captain obvious unterwegs ist\\ 
		Leserichtung & angabe beim Assoziationsnamen\\ 
		Sichtbarkeit & Vor den Rollennamen geschrieben. -private \#protected +public
		$\sim$ package\\ 
		Eigenschaftswert & kann bei Bedarf ans Assoziationsende geschrieben werden.
		Ordned.\\
		Rollenname & bei der Klasse, deren Bedeutung sie beschreibt. Kann zur
		verständlichkeit Beitragen\\
	\end{tabular}

\subsection{Generalisierung}
Beschreibt die Beziehung zwischen einer allgemeinen Klasse (Basisklasse) und
einer spezialisierten Klasse. Die spezialisierte Klasse ist vollständig
konstistent mit der Basisklasse, enthält aber zusätzliche Informationen
(Attribute, Operationen, Assoziationen).\\
	\begin{tabular}{p{3.5cm}p{14.5cm}}
		Vererbung & Unterklasse kann alle eigenschaften als Oberklasse mitbenutzen\\
		Einfachvererbung & \\
		Mehrfachvererbung & \\
		Generalisierung & Die spezialisierte Klasse erweitert die Liste der Attribute,
		Operationen und Assoziationen der Basisklasse\\
		Generalisierungsmenge & spezifiziert, nach welchen Kriterien eine
		Generalisierung modelliert wird\\
	\end{tabular}
	
\subsection{Use-Case Diagramm}
	Ein Use-Case spezifiziert eine Sequenz von Aktionen
	\begin{tabular}{p{3.5cm}p{14.5cm}}
		Akteur & ist eine Rolle, die ein Benutzer des Systems spielt. Jeder Akteur hat
		einen Einfluss uaf das System. Befindet sich stets ausserhalb des Systems\\
	\end{tabular}
