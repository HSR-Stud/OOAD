\section{Begriffe}
\subsection{Klasse}
	Eine Klasse ist eine Vorlage für ein Objekt.\\
	\begin{tabular}{p{3cm}p{15cm}}
		attribut & beschreibt Daten die von den Objekten der Klasse angenommen werden
		können.\\
		& alle Objekte einer Klasse haben dieselben Attribute, können
		aber unterschiedliche Attributwerte haben.\\
		klassenattribut & wenn nur ein
		Attributwert für alle Objekte einer Klasse existiert.\\
		& sie existieren auch, wenn es zu einer Klasse noch keine Objekte gibt.\\
		operation & eine Funktion die auf alle Attributwerte eines Objekts zugriff
		hat\\
		klassenoperation & eine Operation, die der jeweiligen Klasse zugeordnet ist\\
		& kann nicht auf ein einzelnes Objekt der Klasse angewendet werden\\
		verhalten & das Verhalten der Klasse ist die Menge aller operationen\\
	\end{tabular}
\subsection{Objekt}
	Ein Objekt wird aus einer Klasse erzeugt, ist also ein Exemplar einer Klasse.\\
	\begin{tabular}{p{3cm}p{15cm}}
		Zustand & bestimmt durch seine Attributwerte und seine Objektbeziehungen zu
		anderen Objekten\\
		Verhalten (behavior) & die beobachtbaren Effekte aller Operationen\\
		& bestimmt durch die Operationsaufrufe, auf die diese Klasse bzw. deren Objekte
		reagieren.\\
		Objektidendität & jedes Objekt besitzt eine, sind unique\\
	\end{tabular}
\subsection{Paket}
\subsection{Komponente}
\subsection{Assoziation}
\subsection{Generalisierung}
\subsection{Use-Case Diagramm}
\subsection{Aktivitätsdiagramm}
