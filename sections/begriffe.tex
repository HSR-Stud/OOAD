\section{Begriffe}
\subsection{Klasse}
	Eine Klasse ist eine Vorlage für ein Objekt.\\
	\begin{description}[style=multiline,leftmargin=3.5cm,rightmargin=2cm, topsep=0pt]
	\item[Attribut] Beschreibt Daten die von den Objekten der Klasse angenommen
		werden können. Alle Objekte einer Klasse haben dieselben Attribute, können
		aber unterschiedliche Attributwerte haben.
	\item[Klassenattribut] wenn nur ein Attributwert für alle Objekte einer Klasse
		existiert. Sie existieren auch, wenn es zu einer Klasse noch keine Objekte
		gibt.
	\item[Operation] Eine Funktion die auf alle Attributwere eines Objekts Zugriff
		hat.
	\item[Klassenoperation] Eine Operation, die der jeweiligen Klasse zugeordnet
		ist kann nicht auf ein einzelnes Objekt der Klasse angewendet werden. 
	\item[Verhalten] Das Verhlaten der Klasse ist die Menge aller Operationen
	\item[Abstrakte] Von einer abstrakten Klasse können keine Objekte erzeugt
		werden.
	\item[Basisklasse] Vererbt abgeleiteten Klassen Attribute und Operationen
	\end{description}
	
\subsection{Objekt}
	Ein Objekt wird aus einer Klasse erzeugt, ist also ein Exemplar einer Klasse.
	\begin{description}[style=multiline,leftmargin=3.5cm,rightmargin=2cm, topsep=0pt]
		\item[Zustand] bestimmt durch seine Attributwerte und seine
		Objektbeziehungen zu anderen Objekten
		\item[Verhalten (behavior)] die beobachtbaren Effekte aller Operationen
		 bestimmt durch die Operationsaufrufe, auf die diese Klasse bzw. deren Objekte
		reagieren.
		\item[Objektidendität] jedes Objekt besitzt eine, sind unique
	\end{description}
\subsection{Komponente}
Ist ein Softwarebaustein, der über klar definierte Schnittstellen Verhalten
(Funktionalität) bereitstellt.
interface
component

\subsection{Assoziation}
	modelliert Objektbeziehungen zwischen Objekten einer oder mehrerer Klassen. Jede
	Assoziation wird durch Mutliplizitäten, einen optionalen Namen oder Rollennamen
	beschrieben.\\
	\begin{description}[style=multiline,leftmargin=3.5cm,rightmargin=2cm, topsep=0pt]
		\item[ternäre Assoziation] Assoziation zwischen 3 Objekten.
		\item[n-äre Assoziation] zwischen n Objekten
		\item[reflexive Assoziation ] verbindet zwei Objekte einer Klasse
		\item[binäre Assoziation] verbindet zwei Objekte 
		\item[Assoziationsklasse] besitzt die Eigenschaften einer Assoziation und
		einer Klasse 
		\item[Aggregation] Sonderfall der Ass. "`ist Teil von"' oder "`besteht
		aus"'.
		\item[Komposition] besondere Form der Aggregation. Beim Löschen müssen auch
		alle Teile gelöscht werden. Jedes Teil kann nur zu einem Ganzen gehören.
		\item[Navigationsrichtung] das erzeugen des einen erzwingt die Erzeugung des
		anderen
		\item[Multiplizität] bezeichnet die Wichtigkeit der Assoziation. Bezeichnet
		die Anzahl der an der Assoziation beteiligten Objekte.
		\item[abgeleitete Assoziation]
		die Abhängigkeiten sind bereits durch andere Assoziationen beschrieben worden
		\item[Assoziationname] beschreibt im Allgemeinen nur eine Richtung der
		Assoziation. kann fehlen wenn captain obvious unterwegs ist
		\item[Leserichtung] angabe beim Assoziationsnamen
		\item[Sichtbarkeit] Vor den Rollennamen geschrieben. -private \#protected
		+public $\sim$ package
		\item[Eigenschaftswert] kann bei Bedarf ans Assoziationsende geschrieben
		werden. Ordnend.
		\item[Rollenname] bei der Klasse, deren Bedeutung sie beschreibt. Kann zur
		verständlichkeit Beitragen
	\end{description}

\subsection{Generalisierung}
Beschreibt die Beziehung zwischen einer allgemeinen Klasse (Basisklasse) und
einer spezialisierten Klasse. Die spezialisierte Klasse ist vollständig
konstistent mit der Basisklasse, enthält aber zusätzliche Informationen
(Attribute, Operationen, Assoziationen).
	\begin{description}[style=multiline,leftmargin=3.5cm,rightmargin=2cm, topsep=0pt]
		\item[Vererbung] Unterklasse kann alle eigenschaften als Oberklasse
		mitbenutzen
		\item[Einfachvererbung]
		\item[Mehrfachvererbung]
		\item[Generalisierung] Die spezialisierte Klasse erweitert die Liste der
		Attribute, Operationen und Assoziationen der Basisklasse
		\item[Generalisierungsmenge] spezifiziert, nach welchen Kriterien eine
		Generalisierung modelliert wird
	\end{description}
	
\subsection{Use-Case Diagramm}
	Ein Use-Case spezifiziert eine Sequenz von Aktionen
	\begin{description}[style=multiline,leftmargin=3.5cm,rightmargin=2cm, topsep=0pt]
		\item[Akteur] ist eine Rolle, die ein Benutzer des Systems spielt. Jeder
		Akteur hat einen Einfluss uaf das System. Befindet sich stets ausserhalb des Systems\\
	\end{description}
